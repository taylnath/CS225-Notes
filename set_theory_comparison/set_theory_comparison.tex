\documentclass[11pt]{article}
\usepackage[margin=1in, top=15pt]{geometry}
\usepackage{amsmath, amssymb, xcolor, booktabs, graphicx}
\title{CS 225}
\author{{\Large Set Theory and Logic} \\ {\small By Nathan Taylor}}
\date{Fall 2020}

\begin{document}
\maketitle
\hrule

\begin{center}\underline{\bf \huge Comparison of Set Theory and Logic}\end{center}
\bigskip

I find the similarities between set theory and logic to be strikingly accurate:
both have similar identity laws, distributive laws, DeMorgan's laws, etc. 
To me the set-theory versions are more intuitive since you can draw venn-diagram
pictures to reason about the identities, but since we have covered logic first in 
CS225, perhaps the logical rules are more fresh in everyone's mind. In any case, 
I think it is useful to compare the two. 

\bigskip 
When reading through set theory notation, you may want to draw a venn diagram with 
sets $A$ and $B$ contained in some universal set $\mathbb{U}$ to help understand the 
notation. 

\bigskip
Here is a table comparing logical symbols/operations to set theoretic ones:

In the table I will refer to two sets $A$ and $B$ which are both contained 
in some universal set $\mathbb{U}$. You can think of $\mathbb{U}$ as the 
domain of $x$. 

Note: I do not claim that each of these is exactly equivalent! Just a strong 
relation. 

\begin{center}
    {\bf Symbols in Set Theory vs. Logic}

\begin{tabular}{l|l|l|l}
\toprule
Word & Logic & Set Theory & Set Theory Connection  \\
\midrule
Or   &     $\vee$   &         $\cup$    &  $\forall \ x \in \mathbb{U}$, $x \in A\cup B$ if and only if $x \in A$ {\bf or} $x\in B$. \\
And  &     $\wedge$ &         $\cap$    &  $\forall \ x \in \mathbb{U}$, $x \in A \cap B$ if and only if $x \in A$ {\bf and} $x \in B$. \\
Not  &     $\sim$   &   $\phantom{h}^c$ & $\forall \ x \in \mathbb{U}$, $x\in A^c$ if and only if $x$ is {\bf not} in $A$. \\
If / then & $\to$   &  $\subseteq$      & ``$A \subseteq B$'' if and only if ``$\forall \ x \in \mathbb{U}$, {\bf if} $x\in A$ {\bf then} $x\in B$.'' \\
True & $\mathbb{T}$ or T & $\mathbb{U}$ & $\forall \ x\in \mathbb{U}$, $x \in \mathbb{U}$ is always true\\
False & $\mathbb{F}$ or F & $\varnothing$ & $\forall \ x\in \mathbb{U}$, $x$ is not in $\varnothing$. \\
Equivalent & $\equiv$ &  $=$ \ or \   ``if and only if''    \\
\bottomrule
\end{tabular}
\end{center}

\bigskip
\begin{center}\underline{\bf \Large Comparison of Set Theory and Logic Identities}\end{center}

\bigskip
Here is a comparison of set-theoretic versus logical identites (on the next page). Some 
identities (for example the biconditional) don't seem to translate well, 
so I have omitted the set-theory version. In the list, 
$p, q, r$ are propositions, and $A,B, C$ are sets contained in some universal set $\mathbb{U}$. 


\newpage
\newgeometry{top=1in}

\begin{center}
    {\bf Identities in Set Theory vs. Logic}

\begin{tabular}{l|l|l}
\toprule
Logic                    & Set Theory                 & Identity \\
\midrule
\phantom{h} & \phantom{h} & \phantom{h} \\
$\sim (\sim p) \equiv p$ &  $(A^c)^c = A$             & Double Negation \\
\phantom{h} & \phantom{h} & \phantom{h} \\
$p \wedge \mathbb{T} \equiv p$ \ \ \ \ $p \vee \mathbb{F} \equiv p$  
& $A \cap \mathbb{U} = A$ \ \ \ \ $A \cup \varnothing = A$ 
&Identity \\
\phantom{h} & \phantom{h} & \phantom{h} \\
$p \vee \mathbb{T} \equiv \mathbb{T}$ \ \ \ \ $p \wedge \mathbb{F} \equiv \mathbb{F}$ 
& $A \cup \mathbb{U} = \mathbb{U}$ \ \ \ \ $A \cap \mathbb{\varnothing} = \varnothing$ 
&Domination \\
\phantom{h} & \phantom{h} & \phantom{h} \\
$p \wedge p \equiv p$ \ \ \ \ $p \vee p \equiv p$ 
& $A \cup A = A$  \ \ \ \ $A \cap A = A$ 
&Idempotent \\
\phantom{h} & \phantom{h} & \phantom{h} \\
$p \ \vee \sim p \equiv \mathbb{T}$ \ \ \ \ $p \ \wedge \sim p \equiv \mathbb{F}$ 
& $A \cup A^c = \mathbb{U}$ \ \ \ \ $A \cap A^c = \varnothing$
&Negation / Complement \\
\phantom{h} & \phantom{h} & \phantom{h} \\
$p \vee q \equiv q \vee p$ \ \ \ \ $p \wedge q \equiv q \wedge p$ 
&$A \cup B = B \cup A$ \ \ \ \ $A \cap B = B \cap A$
&Commutative \\
\phantom{h} & \phantom{h} & \phantom{h} \\
$(p \vee q) \vee r \equiv p \vee (q \vee r)$ 
& $(A \cup B ) \cup C = A \cup (B \cup C)$
&Associative \\
\phantom{h} & \phantom{h} & \phantom{h} \\
$(p \wedge q) \wedge r \equiv p \wedge (q \wedge r)$ 
&$(A \cap B) \cap C = A \cap (B \cap C)$ 
&Associative \\
\phantom{h} & \phantom{h} & \phantom{h} \\
$p \vee (q \wedge r) \equiv (p \vee q) \wedge (p \vee r)$ 
&$A \cup (B \cap C) = (A \cup B) \cap (A \cup C) $ 
&Distributive \\
\phantom{h} & \phantom{h} & \phantom{h} \\
$p \wedge (q \vee r) \equiv (p \wedge q) \vee (p \wedge r) $
&$A \cap (B \cup C) = (A \cap B) \cup (A \cap C)$ 
& Distributive \\
\phantom{h} & \phantom{h} & \phantom{h} \\
$\sim (p \wedge q) \equiv \ \sim p \ \vee \sim q $
&$(A \cap B)^c = A^c \cup B^c $
&DeMorgan's \\
\phantom{h} & \phantom{h} & \phantom{h} \\
$\sim(p \vee q) \equiv \ \sim p \ \wedge \sim q $
&$(A \cup B)^c = A^c \cap B^c$ 
&DeMorgan's \\
\phantom{h} & \phantom{h} & \phantom{h} \\
$p \vee (p \wedge q) \equiv p $
&$A \cup (A \cap B) = A$ 
&Absorption \\
\phantom{h} & \phantom{h} & \phantom{h} \\
$p \wedge (p \vee q) \equiv p $
&$A \cap (A \cup B) = A$
&Absorption \\
\phantom{h} & \phantom{h} & \phantom{h} \\
\bottomrule
\end{tabular}
\end{center}

In the following table, the set-theoretic analogs were added by me. 
Please do not take my word for these--verify them yourself! 
I do not guarantee that the set-theoretic analogs are correct (although I believe they are). 

\begin{center}
\bigskip
    {\bf Other Identities in Set Theory vs. Logic}

\begin{tabular}{l|l|l}
\toprule
Logic                    & Set Theory                 & Identity \\
\midrule
\phantom{h} & \phantom{h} & \phantom{h} \\
$p \rightarrow q \equiv \ \sim q \rightarrow \ \sim p$ 
& $A \subseteq B$ if and only if $B^c \subseteq A^c$ 
&Contrapositive \\
\phantom{h} & \phantom{h} & \phantom{h} \\
$p \rightarrow q \equiv \ \sim p \vee q$ 
& $A \subseteq B$ if and only if $A^c \cup B = \mathbb{U}$
&Implication \\
\phantom{h} & \phantom{h} & \phantom{h} \\
$(p \rightarrow q) \wedge (p \rightarrow \ \sim q) \equiv \ \sim p$
&If $A \subseteq B$ and $A \subseteq B^c$ then $A = \varnothing$
&Absurdity \\
\phantom{h} & \phantom{h} & \phantom{h} \\
\bottomrule
\end{tabular}
\end{center}

\end{document}