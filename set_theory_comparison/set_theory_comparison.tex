\documentclass[11pt]{article}
\usepackage[margin=1in, top=15pt]{geometry}
\usepackage{amsmath, amssymb, xcolor, booktabs, graphicx}
\title{CS 225}
\author{Set Theory and Logic}
\date{Fall 2020}

\begin{document}
\maketitle
\hrule

\begin{center}\underline{\bf \huge Comparison of Set Theory and Logic}\end{center}
\bigskip

I find the similarities between set theory and logic to be strikingly accurate:
both have similar identity laws, distributive laws, DeMorgan's laws, etc. 
To me the set-theory versions are more intuitive since you can draw venn-diagram
pictures to reason about the identities, but since we have covered logic first in 
CS225, perhaps the logical rules are more fresh in everyone's mind. In any case, 
I think it is useful to compare the two. 

\bigskip
Here is a table comparing logical symbols/operations to set theoretic ones:

In the table I will refer to two sets $A$ and $B$ which are both contained 
in some universal set $\mathbb{U}$. You can think of $\mathbb{U}$ as the 
domain of $x$. 

\begin{center}
\begin{tabular}{lllll}
\toprule
Word & Logic Symbol & Set Theory Symbol & Set Theory Connection  \\
\midrule
Or   &     $\vee$   &         $\cup$    &  $\forall \ x \in \mathbb{U}$, $x \in A\cup B$ if and only if $x \in A$ {\bf or} $x\in B$. \\
And  &     $\wedge$ &         $\cap$    &  $\forall \ x \in \mathbb{U}$, $x \in A \cap B$ if and only if $x \in A$ {\bf and} $x \in B$. \\
Not  &     $\sim$   &   $\phantom{h}^c$ & $\forall \ x \in \mathbb{U}$, $x\in A^c$ if and only if $x$ is {\bf not} in $A$. \\
If / then & $\to$   &  $\subseteq$      & ``$A \subseteq B$'' if and only if ``$\forall \ x \in \mathbb{U}$, {\bf if} $x\in A$ {\bf then} $x\in B$.'' \\
\bottomrule
\end{tabular}
\end{center}


\begin{minipage}[c]{0.4\textwidth}
\underline{\bf Identities:} 
\begin{align*}
    &\sim (\sim p) \equiv p &&\text{Double Negation} \\
    &p \wedge \mathbb{T} \equiv p \ \ \ \ p \vee \mathbb{F} \equiv p &&\text{Identity} \\
    &p \vee \mathbb{T} \equiv \mathbb{T} \ \ \ \ p \wedge \mathbb{F} \equiv \mathbb{F} &&\text{Domination} \\
    &p \wedge p \equiv p \ \ \ \ p \vee p \equiv p &&\text{Idempotent} \\
    &p \ \vee \sim p \equiv \mathbb{T} \ \ \ \ p \ \wedge \sim p \equiv \mathbb{F} &&\text{Negation} \\
    &p \vee q \equiv q \vee p \ \ \ \ p \wedge q \equiv q \wedge p &&\text{Commutative} \\
    &(p \vee q) \vee r \equiv p \vee (q \vee r) &&\text{Associative} \\
    &(p \wedge q) \wedge r \equiv p \wedge (q \wedge r) &&\text{Associative} \\
    &p \vee (q \wedge r) \equiv (p \vee q) \wedge (p \vee r) &&\text{Distributive} \\
    &p \wedge (q \vee r) \equiv (p \wedge q) \vee (p \wedge r) &&\text{Distributive} \\
    &\sim (p \wedge q) \equiv \ \sim p \ \vee \sim q &&\text{DeMorgan's} \\
    &\sim(p \vee q) \equiv \ \sim p \ \wedge \sim q &&\text{DeMorgan's} \\
    &p \vee (p \wedge q) \equiv p &&\text{Absorption} \\
    &p \wedge (p \vee q) \equiv p &&\text{Absorption} \\
    &p \rightarrow q \equiv \ \sim q \rightarrow \ \sim p &&\text{Contrapositive} \\
    &p \oplus q \equiv q \oplus p &&\text{Contrapositive} \\
    &p \rightarrow q \equiv \ \sim p \vee q &&\text{Implication} \\
    &p \leftrightarrow q \equiv (p \rightarrow q) \wedge (q \rightarrow p) &&\text{Biconditional Equivalence} \\
    &(p \wedge q) \rightarrow r \equiv p \rightarrow (q \rightarrow r) &&\text{Exporation} \\
    &(p \rightarrow q) \wedge (p \rightarrow \ \sim q) \equiv \ \sim p &&\text{Absurdity} \\
    &p \vee q \equiv \ \sim p \rightarrow q &&\text{Alternate Implication} \\
    &p \wedge q \equiv \ \sim(p \rightarrow \ \sim q) &&\text{Alternate Implication} \\
    &\sim(p \rightarrow q) \equiv p \ \wedge \sim q &&\text{Alternate Implication} \\
    &\sim \Big( \forall \ x \ P(x) \Big) \equiv \ \exists \ x \ \sim P(x) &&\text{DeMorgan's for Quantifiers} \\
    &\sim \Big( \exists \ x \ Q(x) \Big) \equiv \ \forall \ x \ \sim Q(x) &&\text{DeMorgan's for Quantifiers} 
\end{align*}
\end{minipage}
\newpage
\newgeometry{top=1in}

\begin{center}\underline{\bf \huge Sets}\end{center}
\bigskip

\underline{\bf Symbols:} $\in \ \not\in \ \subseteq \ \subset \ \supseteq \ \supset \ \varnothing \ \cup \ \cap \ \times$
\bigskip

\underline{\bf Common Sets:} 
\begin{align*}
    &\mathbb{N} = \{0, 1, 2, 3, \dots\}  &&\text{natural numbers} \\
    &\mathbb{Z} = \{\dots, -3, -2, -1, 0, 1, 2, 3, \dots \}  &&\text{integers ($\mathbb{Z}$ for German Zahlen, meaning ``numbers")} \\
    &\mathbb{Z^+} = \{1,2,3,\dots\}  &&\text{positive integers} \\
    &\mathbb{Q} = \left\{ \dfrac{p}{q} \ \Big| \ p \in \mathbb{Z}, q \in \mathbb{Z}, q \neq 0\right\}  &&\text{rational numbers} \\
    &\mathbb{U} = \{*\}  &&\text{universal set}
\end{align*}
\bigskip

\underline{\bf Identities:}
\begin{align*}
    &A \cup \varnothing = A \ \ \ \ A \cap \mathbb{U} = A &&\text{Identity} \\
    &A \cup \mathbb{U} = \mathbb{U} \ \ \ \ A \cap \mathbb{\varnothing} = \varnothing &&\text{Domination} \\
    &A \cup A = A \ \ \ \ A \cap A = A &&\text{Idempotent} \\
    &A \cup A^c = \mathbb{U} \ \ \ \ A \cap A^c = \varnothing &&\text{Complement} \\
    &A \cup B = B \cup A \ \ \ \ A \cap B = B \cap A &&\text{Commutative} \\
    &(A \cup B ) \cup C = A \cup (B \cup C) &&\text{Associative} \\
    &(A \cap B) \cap C = A \cap (B \cap C) &&\text{Associative} \\
    &A \cap (B \cup C) = (A \cap B) \cup (A \cap C) &&\text{Distributive} \\
    &A \cup (B \cap C) = (A \cup B) \cap (A \cup C) &&\text{Distributive} \\
    &(A \cup B)^c = A^c \cap B^c &&\text{DeMorgan's} \\
    &(A \cap B)^c = A^c \cup B^c &&\text{DeMorgan's} \\
    &A \cup (A \cap B) = A &&\text{Absorption} \\
    &A \cap (A \cup B) = A &&\text{Absorption}
\end{align*}
\bigskip
\underline{\bf Symbols:} $\geq \ \leq \ \neq \ \neg \ \sim \ \wedge \ \vee \ \oplus \ \equiv \ \rightarrow \ \leftrightarrow \ \square \ \exists \ \forall$
\bigskip
\bigskip
\begin{center}\underline{\bf Sum Formulas:} \end{center}
\begin{center}
\scalebox{1.4}{
}
\end{center}
\end{document}