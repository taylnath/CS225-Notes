\documentclass[11pt]{article}
\usepackage[margin=1in, top=15pt]{geometry}
\usepackage{amsmath, amssymb, amsthm, xcolor, booktabs, graphicx}
\usepackage{enumitem}
\title{CS 225}
\author{Theorems from Chapter 4}
\date{Fall 2020}

\begin{document}
\maketitle
\hrule

\section*{From 4.1:}
\subsection*{Assumptions:}
\begin{itemize}
    \item Basic laws of algebra from appendix $A$ (I assume we do not have to cite these--I listed some on
    the last page of this document)
    \item Properties of equality: $x = x$, $x = y \implies y = x$, $x = y \text{ and } y = z \implies x = z$
    \item Principle of substitution: if $x = y$, then we can substitute in $y$ wherever $x$ appears. 
    \item The integers are closed under addition, subtraction and multiplication. (The book also 
    mentions that there is no integer between 0 and 1--does that need to be an assumption??)
\end{itemize}

\subsection*{Definitions:}
\begin{itemize}
    \item An integer $n$ is {\bf even} if and only if 
    it can be written as $n = 2k$ for some integer $k$. 
    \item An integer $n$ is {\bf odd} if and only if 
    it can be written as $n = 2k + 1$ for some integer $k$. 
    \item An integer $n$ is {\bf prime} if and only if 
    $n > 1$ and for all positive integers $r$ and $s$, if 
    $n = rs$, then either $r$ or $s$ equals $n$. 
    \item An integer $n$ is {\bf composite} if and only if
    $n > 1$ and $n = rs$ for some integers $r$ and $s$ with
    $1< r< n$ and $1< s< n$. 
\end{itemize}

\subsection*{Theorem 4.1.1}
The sum of any two even integers is even. 

\bigskip
\hrule
\bigskip

\section*{From 4.2:}
\subsection*{Theorem 4.2.1:}
The difference of any odd integer and any even integer is odd. 

\bigskip
\hrule
\bigskip

\newpage
\newgeometry{top=1in}

\section*{From 4.3:}
\subsection*{Definition:}
A real number $r$ is {\bf rational} if and only if 
there are integers $a$ and $b$ with $b \neq 0$ so that
$r = \dfrac{a}{b}$

\subsection*{Property:}
The zero product property is: if neither of two 
real numbers is zero, then their product is also not zero. (
    this is T11 in appendix A)

\subsection*{Theorem 4.3.1:}
Every integer is a rational number.

\subsection*{Theorem 4.3.2:}
The sum of any two rational numbers is a rational number. 

\subsection*{Corollary 4.2.3:}
The double of a rational number is a rational number. 

\subsection*{Result of Exercise 12:}
The square of any rational number is a rational number. 

\subsection*{Result of Exercise 13:}
The negative of any rational number is a rational number. 

\subsection*{Result of Exercise 14:}
The cube of any rational number is a rational number. 

\subsection*{Result of Exercise 15:}
The product of any two rational numbers is a rational number. 

\subsection*{Result of Exercise 17:}
The difference of any two rational numbers is a rational number. 

\newpage
\section*{From Appendix A:}
\begin{itemize}
    \item Field Axioms for real numbers:
    \begin{enumerate}[label=F\arabic*]
        \item Commutative laws for addition and multiplication
        \item Associative laws for addition and multiplication
        \item Distributive laws
        \item Existence of an identity for addition and multiplication
        \item Existence of additive inverses (negative numbers)
        \item Existence of multiplicative inverses (reciprocals)
    \end{enumerate}
    \item Other theorems:
    \begin{enumerate}[label=T\arabic*]
        \item Cancellation law for addition: $a + b = a + c \implies b = c$
        \item Possibility of subtraction: There is a unique solution to $a + x = b$ (namely $x = b - a$)
        \item $b - a = b + (-a)$
        \item $-(-a) = a$
        \item $a(b - c) = ab - ac$
        \item $0\cdot a = a\cdot 0 = 0$
        \item Cancellation law for multiplication: $ab = ac \text{ and } a \neq 0 \implies b = c$
        \item Possibility of division: If $a \neq 0$, there is a unique solution to $ax = b$. (namely $x = b/a$)
        \item If $a \neq 0$, $\dfrac{b}{a} = b \cdot a^{-1}$
        \item If $a \neq 0$, $(a^{-1})^{-1} = a$
        \item Zero product property: If $ab = 0$, then $a = 0$ or $b = 0$
        \item $(-a) b = a(-b) = - (ab)$, $(-a)(-b) = ab$, $$-\dfrac{a}{b} = \dfrac{-a}{b} = \dfrac{a}{-b}$$
        \item Equivalent fractions property: $\dfrac{a}{b} = \dfrac{ac}{bc}$ if $b\neq 0$ and $c\neq 0$
        \item Fraction addition: $\dfrac{a}{b} + \dfrac{c}{d} = \dfrac{ad + bc}{bd}$ if $b\neq 0$ and $d\neq 0$
        \item Multiplying fractions: $\dfrac{a}{b} \cdot \dfrac{c}{d} = \dfrac{ac}{bd}$ 
        \item Dividing fractions: $\dfrac{a/b}{c/d} = \dfrac{a}{b} \cdot \dfrac{d}{c}$
    \end{enumerate}
    \item Order Axioms (see the appendix on page A-2)
    \item Order Theorems (see the appendix A-3)
    \item Least upper bound axiom (see the appendix A-3)
\end{itemize}
\end{document}